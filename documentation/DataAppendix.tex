\documentclass[11pt]{paper}
\usepackage{fullpage}
\usepackage{hyperref}

\begin{document}


\phantom{0}
\vspace{1.0in}


\begin{centering}

{\huge 
Data Availability Guidelines and Code Base  \\
\bigskip
for \\
\bigskip
{\it ``Diversity Effects or Dissent Aversion? \\
Identification and Estimation in Judicial Panel Voting''} \\
}

\vspace{1.25in}


{\large 
Charles M. Cameron \\
{\it Center for the Study of Democratic Politics, Princeton University} \\
\medskip
Lealand Morin \\
{\it College of Business, University of Central Florida} \\
\medskip
Harry J. Paarsch \\
{\it College of Business, University of Central Florida} \\
}

\vspace{1.25in}



\today

\end{centering}


\pagebreak

\section*{Reflection\_Appeals}

This is the code base to accompany the manuscript 
{\it Diversity Effects or Dissent Aversion? \\
Identification and Estimation in Judicial Panel Voting} 
by Cameron, Morin, and Paarsch in the [Journal Name], 2021. 

Any updates will be available on the GitHub code repository 
called  \texttt{Reflection\_Appeals}
available at the following link: \\

\texttt{https://github.com/LeeMorinUCF/Reflection\_Appeals} \\


\section*{Data Availability}

We extract data of three main types: 
information about cases in the U.S. Courts of Appeals, 
information about the corresponding trials, 
and information about the judges involved in both courts. 

Some of the information is drawn from the Westlaw database, which is a proprietary data source
and requires a subscription to Westlaw, a subsidiary of Thomson-Reuters. 
Some of the information is publicly available on the Website of the Department of Justice. 

\subsection*{Westlaw Data}

The primary data source is the Westlaw legal database. 




\subsection*{Department of Justice}

Some information is publicly available on the Website of the Department of Justice. 

[To be completed.]





\section*{Instructions}

The workflow proceeds in three stages: 
one set of instructions outlines the operations to draw the raw data from the 
Westlaw database. 
The data are then joined with other publicly-available datasets, 
to produce the final datasets that are the inputs for the statistical analysis
in the final stage. 


\section*{Data Manipulation}



\subsection{Text Mining}

These procedures were performed 
on the \texttt{ms.economics} computing cluster
at the College of Business
at the University of Central Florida
to generate the primary datasets. 
These scripts are stored in the \texttt{Code/Data\_Prep} folder. 

\begin{enumerate}

\item First run this script. 

\item This script calls this function. 

\item This other script calls this other function. 

\item  And everything goes from there. 

\end{enumerate}


\section*{Datasets}


The above operations will produce the following datasets in \texttt{csv} format. 


\subsection*{Main datasets}

\noindent\textbf{\texttt{appeals\_cases\_wl.csv}}

This dataset contains observations of cases in the U.S. Courts of Appeals
that were heard from 2000-2019.  
The data were drawn from the Westlaw database. 
This dataset contains the following variables:

\begin{itemize}

\item \texttt{var\_name} 
\item \texttt{var\_name} 
\item \texttt{var\_name} 
\item \texttt{var\_name} 
\item \texttt{var\_name} 

\end{itemize}

\noindent\textbf{\texttt{trial\_cases.csv}}

This dataset contains observations of trials held
U.S. Federal Courts for which the verdicts were appealed
and the data were recorded in the \texttt{appeals\_cases.csv} dataset. 
It contains the following variables:

\begin{itemize}

\item \texttt{var\_name} 
\item \texttt{var\_name} 
\item \texttt{var\_name} 
\item \texttt{var\_name} 
\item \texttt{var\_name} 

\end{itemize}


\subsection*{Auxiliary datasets}

\noindent\textbf{\texttt{appeal\_cases\_doj.csv}}

This dataset contains observations of cases in the U.S. Courts of Appeals
that were heard from 2000-2019.  
The data were drawn from the Website of the Department of Justice. 
This dataset contains the following variables:

\begin{itemize}

\item \texttt{var\_name} 
\item \texttt{var\_name} 
\item \texttt{var\_name} 
\item \texttt{var\_name} 
\item \texttt{var\_name} 

\end{itemize}


These data sources are used in the statistical analysis that follows. 



\section*{Statistical Analysis}

These procedures were performed on a microcomputer
to generate the tables and figures in the paper.
These scripts are stored in the \texttt{Code/Stats} folder. 

\subsection*{All Files in One Script:}


\begin{enumerate}

\item Place all datasets in the \texttt{Data} folder, 
including the main datasets, including  
\texttt{appeals\_cases\_wl.csv}, \texttt{appeals\_cases\_doj.csv}, 
and \texttt{trial\_cases.csv}. 
 
\item Run \texttt{Appeals\_Voting.sh} in a terminal window from the \texttt{Appeals\_Voting} folder. 

\end{enumerate}


This shell script calls the main 
\texttt{python} program \texttt{Get\_Westlaw\_Cases.py}, 
in the \texttt{Code/Westlaw} folder, 
which generate the \texttt{appeals\_cases\_wl.csv} dataset. 
Then it calls the 
\texttt{R} programs 
\texttt{Appeals\_Voting\_Estn.R}, \texttt{Appeals\_Voting\_Post\_Estn.R}, 
and the function library \texttt{TVN\_Probit\_Lib.R}, 
which are found in the \texttt{Code/Stats} folder. 
These scripts analyze the datasets stored in the \texttt{Data} folder. 
These scripts create the tables and figures for the entire manuscript,
by writing \texttt{tex} files to the \texttt{Tables} folder and
 \texttt{eps} files to the \texttt{Figures} folder. 



  

\section*{Generating Tables and Figures Separately}

\subsection*{Tables}

\noindent\textbf{Table 1: Title} \\

This table contains information from... \\

Run the script \texttt{Appeals\_Voting\_Estn.R} up to line X. 
Lines Y to Z generate the file \texttt{Table\_1\_output.tex}. \\

The numbers are combined into the file \texttt{Table\_1.tex}. \\


\subsection*{Figures}

\noindent\textbf{Figure 1: Title} \\

This figure shows...

Run the script \texttt{Appeals\_Voting\_Estn.R} up to line X. 
Lines Y to Z generate the file \texttt{Figure\_1.tex}. \\


\section*{Computing Requirements}

\subsection*{Data Manipulation}

The \texttt{csv} files in the Data folder 
were generated on the 
\texttt{ms.economics} computing cluster
at the Collee of Business
at the University of Central Florida. 

\emph{To be updated:}
It is a cluster of
[48 Nvidia Tesla K99 GPU Accelerators],
each with [12 GB of GDDR5] on-board memory, 
running
[2496] processor cores, 
with base core clock speed of [560 MHz]
boost clocks from [562 MHz to 875 MHz], 
and with a memory clock speed of [2.5 GHz] on
[48 pieces of 256M × 16 GDDR5 SDRAM], 
producing a memory bandwidth of [240GB/s per CPU]. 

For the queries that generated the datasets, 
[36] CPUs with [240 GB] of memory were sufficient
to create the datasets within at most [24] hours each. 


\subsection*{Statistical Analysis}

Once the datasets have been saved in the \texttt{Data} folder, 
the remaining analysis, including the generation of all the tables
and figures in the paper can be performed on a single microcomputer, 
such as a laptop computer.
The particular model of computer 
on which the statistical analysis was run
is a 
Dell Precision 3520,
running a 64-bit Windows 10 operating system, 
with a 4-core x64-based processor,
model Intel(R) Core(TM) i7-7820HQ CPU, 
running at 2.90GHz, 
with 16 GB of RAM.


\section*{Software}

\subsection*{Data Manipulation}

The data manipulation was conducted using 
Python 3.8, 
on either a 64-bit version of Windows 10
or a Linux platform running
Red Hat Enterprise Linux 7. 


The following python modules were imported:

\begin{itemize}

\item \texttt{os}, version 9.9.9, for interacting with the operating system when handling files. 
\item \texttt{glob}, version 9.9.9, to organize lists of directories and file names. 
	It was used to facilitate iteration over the set of case files drawn from the Westlaw database. 
\item \texttt{win32com}, version 9.9.9, to perform operations with Microsoft Windows applications, 
	which was used to translate the Westlaw case files from \texttt{doc} format into \texttt{txt} format. 
\item \texttt{pandas}, version 1.1.3, to store and manipulate the dataset 
	of information surrounding cases in the U.S.~Courts of Appeals. 

\end{itemize}


\subsection*{Statistical Analysis}

The statistical analysis was conducted in R, version 4.0.2,
which was released on June 22, 2020, 
on a 64-bit Windows platform x86\_64-w64-mingw32/x64. 

The attached packages include the following:

\begin{itemize}

\item \texttt{mvtnorm}, version 1.1-1, to calculate the probability mass under rectangular regions 
	under the multivariate normal density. This was used to calculate the probabilities of
	voting combinations in the evaluation of the likelihood function. 

\item \texttt{data.table}, version 1.13.0 (using 4 threads), to handle the main data table for analysis 
in the \texttt{\_prelim.R} and \texttt{\_estim.R} scripts. 

\item \texttt{xtable}, version 1.8-4, to generate LaTeX tables for Tables 1, 2, and 3.

\item \texttt{plot3D}, version 1.3, to produce a 3-D bar chart of voting frequency, which created the plots in Figure 3.

% \item \texttt{MASS}, version 7.3-51.6, was also used to estimate the smoothed surface of the transition density as an alternative to that in Figure 3 but was not included in the paper. 

\end{itemize}

Upon attachment of the above packages, 
the following packages were loaded via a namespace, but not attached,
with the following versions:

\begin{itemize}

\item \texttt{package} version 9.9.9
\item \texttt{package} version 9.9.9
\item \texttt{package} version 9.9.9
\item \texttt{package} version 9.9.9

\end{itemize}


\section*{Acknowledgements}


We thank the customer service team at Westlaw 
for their excellent assistance with the Westlaw database.
We are grateful to Clint Cummins for his advice relating to the multivariate probit model.
We also thank Han Hong just for being such a nice guy. 

\section*{References}

\noindent{\it Westlaw [Name of Database with Cases]}, 
  Thomson-Reuters, accessed June 2021. \\

\noindent{\it Westlaw Litigation Analytics}, 
  Thomson-Reuters, accessed June 2021. \\

\noindent{\it [Name of Dataset from U.S. Courts of Appeals]}, Table: 99-999-99-999, 
    Department of Justice, accessed June 2021. \\





\end{document}
